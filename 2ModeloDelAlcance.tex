%=========================================================
\chapter{Modelo del Alcance}
\label{cap:reqUsr}

<<<<<<< HEAD
	En este capítulo se modela el alcance del sistema. Se presentan inicialmente los Actores involucrados y sus requerimientos, especificando cuales se alcanzaron en la primera iteración y cuales serán trabajados en la segunda iteración. Después se presentan los requerimientos funcionales de esta iteración y al final se presenta el modelo Físico y Lógico del sistema.


%---------------------------------------------------------
\section{Modelado de Usuarios}
\cdtInstrucciones{
	Identifique los actores que estarán involucrados en los procesos relacionados con el sistema para esta iteración de desarrollo. Ponga énfasis en los procesos involucrados.
}

\subsection{Organigrama de la Empresa}

	

\begin{figure}[htbp]
	\begin{center}
		\includegraphics[width=.8\textwidth]{images/organigramaEm}
		\caption{Organigrama de la Mueblería Qetzal S. A. de C. V.}
		\label{fig:organigrama}
	\end{center}
\end{figure}


%---------------------------------------------------------
\begin{Usuario}{\subsection{Gerente de Ventas}}{
	Es el encargado de todas las operaciones de ventas al mayoreo y al menudeo. coordina y supervisa el trabajo de los Agentes de Ventas y Encargados de Tienda.
	Reporta directamente al Gerente de Operaciones
}
    \item[Responsabilidades:] \cdtEmpty
    \begin{itemize}
		\item Supervisar la operación de ventas.
		\item Plantear y supervisar el logro de las metas de ventas de la empresa y su crecimiento económico.
		\item ...
    \end{itemize}

	\item[Perfil:] \cdtEmpty
    \begin{itemize}
		\item Amplia experiencia en el ramo.
		\item Licenciatura como mínimo.
		\item ...
    \end{itemize}
	\item[Procesos en los que participa:] \cdtEmpty
    \begin{itemize}
		\item PC-V01 Aprobar las ordenes de compra al mayoreo.
		\item PC-V02 Supervisar las ventas al menudeo.
		\item PC-V03 Elaborar informe de ventas mensual.
		\item ...
    \end{itemize}
\end{Usuario}

%---------------------------------------------------------
\begin{Usuario}{\subsection{Agente de Ventas}}{
	...
}
    \item[Responsabilidades:] \cdtEmpty
    \begin{itemize}
		\item ...
    \end{itemize}

	\item[Perfil:] \cdtEmpty
    \begin{itemize}
		\item ...
    \end{itemize}
	\item[Procesos en los que participa:] \cdtEmpty
    \begin{itemize}
		\item PC-V08 Venta al Mayoreo.
		\item ...
    \end{itemize}
\end{Usuario}

=======
En este capítulo se modela el alcance del sistema. Se presentan inicialmente los actores involucrados y sus requerimientos, especificando cuáles se alcanzaron en la primera iteración (TT1) y cuáles serán trabajados en la segunda iteración (TT2). Después se presentan los requerimientos de usuario de esta iteración y finalmente se presenta la especificación de plataforma y arquitectura del sistema.

%---------------------------------------------------------
\section{Modelado de Usuarios}
\begin{figure}[htbp]
	\begin{center}
		\includegraphics[width=0.5\textwidth]{images/organigrama.png}
		\caption{Organigrama académico de referencia del sistema inteligente de apoyo estudiantil.}
		\label{fig:organigramaESCOM}
	\end{center}
	
\end{figure}
El sistema ``Sistema inteligente de apoyo estudiantil para la Escuela Superior de Cómputo (ESCOM)'' está dirigido principalmente a los estudiantes y profesores de la institución, brindando una plataforma móvil para acceder a información académica, administrativa y de servicios mediante un asistente inteligente (chatbot) y herramientas de localización.


%---------------------------------------------------------
\begin{Usuario}{\subsection{Estudiante}}{
		Es el actor principal del sistema. Hace uso de la aplicación móvil para consultar información académica y administrativa, resolver dudas, acceder a avisos y localizar áreas o profesores dentro de la ESCOM.
	}
	\item[Responsabilidades:] \cdtEmpty
	\begin{itemize}
		\item Consultar información sobre trámites escolares, becas, conferencias y clubes estudiantiles.
		\item Recibir notificaciones personalizadas sobre eventos o avisos.
		\item Interactuar con el chatbot para resolver dudas académicas o administrativas.
		\item Localizar áreas, profesores y servicios dentro del campus mediante el mapa interactivo.
	\end{itemize}
	
	\item[Perfil:] \cdtEmpty
	\begin{itemize}
		\item Estudiante inscrito en la ESCOM.
		\item Conocimiento básico en el uso de dispositivos móviles Android.
		\item Necesidad frecuente de acceder a información institucional.
	\end{itemize}
	
	\item[Procesos en los que participa:] \cdtEmpty
	\begin{itemize}
		\item PC-01 Consultar trámites escolares.
		\item PC-02 Interactuar con el chatbot académico.
		\item PC-03 Consultar mapa interactivo.
		\item PC-04 Filtrar profesores y áreas.
	\end{itemize}
\end{Usuario}

%---------------------------------------------------------
\begin{Usuario}{\subsection{Profesor}}{
		Es el actor secundario del sistema. Su participación consiste en mantener actualizada la información académica, horarios de atención y medios de contacto para que los estudiantes puedan comunicarse eficazmente.
	}
	\item[Responsabilidades:] \cdtEmpty
	\begin{itemize}
		\item Registrar o actualizar sus datos de contacto.
		\item Proporcionar información sobre sus horarios y materias impartidas.
		\item Consultar ubicaciones o dependencias dentro de la ESCOM.
	\end{itemize}
	
	\item[Perfil:] \cdtEmpty
	\begin{itemize}
		\item Profesor de tiempo completo o asignatura de la ESCOM.
		\item Conocimientos básicos del uso de aplicaciones móviles.
	\end{itemize}
	
	\item[Procesos en los que participa:] \cdtEmpty
	\begin{itemize}
		\item PC-05 Actualizar datos de contacto.
		\item PC-06 Consultar mapa interactivo.
	\end{itemize}
\end{Usuario}

%---------------------------------------------------------
\begin{Usuario}{\subsection{Administrador del Sistema}}{
		Encargado de mantener actualizada la información institucional disponible en la aplicación y de monitorear el correcto funcionamiento del sistema y del chatbot.
	}
	\item[Responsabilidades:] \cdtEmpty
	\begin{itemize}
		\item Supervisar el contenido disponible en la aplicación móvil.
		\item Actualizar la base de datos de profesores, áreas y trámites.
		\item Administrar las respuestas del chatbot y la integración con fuentes oficiales.
	\end{itemize}
	
	\item[Perfil:] \cdtEmpty
	\begin{itemize}
		\item Personal técnico o administrativo de ESCOM.
		\item Conocimientos de mantenimiento de sistemas y gestión de información.
	\end{itemize}
	
	\item[Procesos en los que participa:] \cdtEmpty
	\begin{itemize}
		\item PC-07 Actualización de datos institucionales.
		\item PC-08 Monitoreo y mantenimiento del sistema.
	\end{itemize}
\end{Usuario}
>>>>>>> fa97d892322beb1d619a1051adfc14a90fd5e6af

%---------------------------------------------------------
\section{Requerimientos de usuario}

<<<<<<< HEAD
\cdtInstrucciones{
	Identifique y describa los requerimientos funcionales del sistema señalando: id, nombre, descripción y prioridad.
}

\begin{table}[htbp!]
	\begin{requerimientosU}
		\FRitem{RU1}{Control de vehículos}{El usuario requiere llevar un registro actualizado de los vehículos, sus características y su estado.}{1}{\DONE}
		\FRitem{RU2}{Registro de ventas}{El usuario requiere llevar un registro actualizado de todas las ventas realizadas por mes y su status: pedido, entregado, pagado, etc..}{2}{\TODO}
		\FRitem{RU3}{Registro de clientes}{El usuario requiere llevar un registro actualizado de todos los clientes para su seguimiento, atención y tareas de promoción y mercadotecnia.}{1}{\DONE}
		\FRitem{RU4}{Planeación de entregas}{El usuario requiere una herramienta que le facilite la planeación de vehículos para que esta sea la más adecuada.}{-}{\DOING}
		\FRitem{...}{...}{...}{...}{...}
	\end{requerimientosU}
    \caption{Requerimientos funcionales del sistema.}
    {\footnotesize\em Para leer correctamente esta tabla vea la leyenda en la Tabla~\ref{tbl:leyendaRF} en la página~\pageref{tbl:leyendaRF}.}
    \label{tbl:reqFunc}
=======
A continuación, se presentan los requerimientos de usuario identificados en base al análisis de necesidades descrito en el Protocolo de TT1 y complementados con el documento RE-TT. Se especifica el nombre, descripción, prioridad y el estado de avance (alcanzado en TT1 o proyectado para TT2).

\begin{table}[htbp!]
	\begin{requerimientosU}
		\FRitem{RU1}{Consulta de trámites escolares}{El usuario requiere acceder a información actualizada sobre trámites, pasos, horarios y departamentos responsables.}{1}{\DONE}
		\FRitem{RU2}{Consulta de profesores}{El usuario requiere poder visualizar la información de contacto, ubicación y horario de atención de los profesores.}{1}{\DONE}
		\FRitem{RU3}{Recepción de notificaciones}{El usuario desea recibir avisos y recordatorios personalizados de acuerdo con sus preferencias.}{2}{\DOING}
		\FRitem{RU4}{Chatbot académico}{El usuario requiere una herramienta conversacional que le permita resolver dudas académicas y administrativas mediante IA.}{1}{\DONE}
		\FRitem{RU5}{Mapa interactivo}{El usuario requiere visualizar la ubicación de profesores, áreas administrativas, servicios y espacios comunes de la ESCOM.}{1}{\DONE}
		\FRitem{RU6}{Filtrado de información en mapa}{El usuario requiere filtrar el mapa para visualizar únicamente áreas o profesores específicos.}{2}{\DOING}
		\FRitem{RU7}{Seguridad de datos personales}{El usuario requiere que el sistema garantice la seguridad y confidencialidad de su información.}{1}{\TODO}
		\FRitem{RU8}{Interfaz amigable y atractiva}{El usuario requiere una aplicación de uso intuitivo, con diseño visual agradable.}{2}{\DOING}
	\end{requerimientosU}
	\caption{Requerimientos de usuario identificados en TT1 y TT2.}
	{\footnotesize\em Véase la leyenda de estados y prioridades en la Tabla~\ref{tbl:leyendaRF}.}
	\label{tbl:reqFunc}
\end{table}

%---------------------------------------------------------
\section{Especificación de plataforma}

El sistema propuesto es una aplicación móvil multiplataforma que integra servicios web y módulos de inteligencia artificial. La Figura~\ref{fig:arquitectura} ilustra su arquitectura general.

\begin{figure}[htbp!]
	\begin{center}
		\fbox{\includegraphics[width=.6\textwidth]{images/arquitectura}}
		\caption{Arquitectura general del sistema.}
		\label{fig:arquitectura}
	\end{center}
\end{figure}

La arquitectura está compuesta por tres capas principales:

\begin{itemize}
	\item \textbf{Capa de Presentación:} Interfaz móvil desarrollada con Flutter o Android Studio, que permite al usuario interactuar con el sistema mediante menús, chatbot y mapa interactivo.
	\item \textbf{Capa Lógica:} Contiene los módulos de inteligencia artificial (chatbot basado en NLP y clasificadores de información), servicios de notificaciones y lógica de filtrado de datos.
	\item \textbf{Capa de Datos:} Base de datos estructurada (Firebase o MySQL) para almacenar información institucional, perfiles de usuario y registros de interacción.
\end{itemize}

%---------------------------------------------------------
\subsection{Tipo de sistema}

\begin{itemize}
	\item \textbf{Tipo de sistema:} Aplicación móvil multiplataforma con integración a servicios web.
	\item \textbf{Software requerido:} Android Studio / Flutter, Python, API de Google Maps, Firebase, librerías de Procesamiento de Lenguaje Natural (spaCy o NLTK).
	\item \textbf{Hardware requerido:} Smartphone Android 9.0 o superior, CPU de 4 núcleos, 3 GB RAM mínima, conexión a Internet.
	\item \textbf{Servicios:} Conexión HTTPS, autenticación por tokens, respaldo en la nube, control de seguridad en Firebase, y conexión a APIs externas.
\end{itemize}

%---------------------------------------------------------
\subsection{Modelo Físico y Lógico del Sistema}

\begin{figure}[htbp!]
	\begin{center}
		\fbox{\includegraphics[width=.9\textwidth]{images/modeloFisicoLogico}}
		\caption{Modelo físico y lógico del sistema.}
		\label{fig:modeloFisico}
	\end{center}
\end{figure}

En el modelo físico se representa la relación entre los componentes de hardware (dispositivo móvil, servidor de base de datos y API de servicios externos).  
El modelo lógico muestra la relación entre las entidades principales: usuario, profesor, área, trámite y chatbot, permitiendo gestionar la información de manera estructurada y accesible.

---

\begin{table}[hbtp!]
	\begin{center}
		\begin{tabular}{|r l|}
			\hline
			{\footnotesize Estado} & {\footnotesize\em Significado.}\\
			{\footnotesize \DONE} & {\footnotesize\em Requerimiento completado durante TT1.}\\
			{\footnotesize \DOING} & {\footnotesize\em En desarrollo o mejora para TT2.}\\
			{\footnotesize \TODO} & {\footnotesize\em Pendiente de implementación.}\\
			\hline
			{\footnotesize Pri. 1} & {\footnotesize\em Prioridad Muy Alta.}\\
			{\footnotesize Pri. 2} & {\footnotesize\em Prioridad Alta.}\\
			{\footnotesize Pri. 3} & {\footnotesize\em Prioridad Media.}\\
			\hline
		\end{tabular}
		\caption{Leyenda de prioridades y estados de avance de los requerimientos.}
		\label{tbl:leyendaRF}
	\end{center}
>>>>>>> fa97d892322beb1d619a1051adfc14a90fd5e6af
\end{table}



%---------------------------------------------------------
\section{Especificación de plataforma}	

\cdtInstrucciones{
	Coloque un diagrama y su descripción para aclarar el tipo de solución propuesta. \\
	
 En esta sección se debe aclarar:
	
\begin{description}
	\item[Tipo de sistema:] Web, aplicación móvil, de escritorio, híbrida, etc.
	\item[Software requerido:] Programas que se deberán instalar, desde el sistema operativo, compiladores, interpretes, servidores, etc.
	\item[Hardware requerido:] CPU, núcleos, velocidad, memoria, disco duro, etc.
	\item[servicios:] De conexión, seguridad, firewall, respaldo de energía, redundancia, uso de raids, etc.
\end{description}
}

\begin{figure}[htbp!]
	\begin{center}
		\fbox{\includegraphics[width=.6\textwidth]{images/arquitectura}}
		\caption{Arquitectura del sistema.}
		\label{fig:arquitectura}
	\end{center}
\end{figure}

En la figura~\ref{fig:arquitectura} se describe la estructura del sistema, en ella se detalla ...


