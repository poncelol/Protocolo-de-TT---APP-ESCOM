%=========================================================
\chapter{Introducción}

<<<<<<< HEAD

\cdtInstrucciones{
	Presentar el documento, indicando su contenido, a quien va dirigido, quien lo realizó, por que razón, dónde y cuando. \\
}
	Este documento contiene la Especificacion del ptoyecto ``{\em Nombre del proyecto}'' correspondiente al trabajo realizado en el semestre 2016-2017-2 para la materia de Análisis y diseño orientado a objetos en el grupo 2CV9 por el equipo {\em Nombre del equipo}.
=======
Este documento contiene la especificación del proyecto ``{\em Sistema inteligente de apoyo estudiantil para la Escuela Superior de Cómputo (ESCOM)}'' correspondiente al trabajo realizado durante el periodo 2025-1 para la materia de Trabajo Terminal por el equipo conformado por {\em Rivero López Valeria} y {\em Ponce Fragoso Emmanuel}, bajo la dirección del profesor {\em Ulises Saldaña Vélez}, en la {\em Escuela Superior de Cómputo (ESCOM)} del Instituto Politécnico Nacional.
>>>>>>> fa97d892322beb1d619a1051adfc14a90fd5e6af

%---------------------------------------------------------
\section{Presentación}

<<<<<<< HEAD
	Este documento contiene información teórica sobre el proyecto que se está desarrollando, fundamentando la importancia que tiene, mostrando los requerimientos del usuario y del sistema, planteamiento del problema, desarrollo, resultados y conclusiones que hemos obtenido durante esta primera parte. Tiene como objetivo establecer las funciones y el desarrollo que se llevará a cabo para obtener el resultado final. 
	un reporte técnico del proyecto ``{\em ESCOMobile}''.
	
%---------------------------------------------------------
\section{Organización del contenido}

	En el capítulo \ref{cap:reqUsr} ...
	
	En el capítulo \ref{cap:reqSist} ...

%---------------------------------------------------------
\section{Notación, símbolos y convenciones utilizadas}

	Los requerimientos funcionales utilizan una clave RFX, donde:
	
\begin{description}
	\item[X] Es un número consecutivo: 1, 2, 3, ...
	\item[RF] Es la clave para todos los {\bf R}equerimientos {\bf F}uncionales.
\end{description}

	Los requerimientos del usuario utilizan una clave RUX, donde:
	
\begin{description}
	\item[X] Es un número consecutivo: 1, 2, 3, ...
	\item[RU] Es la clave para todos los {\bf R}equerimientos del {\bf U}suario.
\end{description}

	Además, para los requerimeitnos funcionales se usan las abreviaciones que se muestran en la tabla~\ref{tbl:leyendaRF}.
\begin{table}[hbtp!]
	\begin{center}
    \begin{tabular}{|r l|}
	    \hline
    	{\footnotesize Id} & {\footnotesize\em Identificador del requerimiento.}\\
    	{\footnotesize Pri.} & {\footnotesize\em Prioridad}\\
    	{\footnotesize Ref.} & {\footnotesize\em Referencia a los Requerimientos de usuario.}\\
    	{\footnotesize MA} & {\footnotesize\em Prioridad Muy Alta.}\\
    	{\footnotesize A} & {\footnotesize\em Prioridad Alta.}\\
    	{\footnotesize M} & {\footnotesize\em Prioridad Media.}\\
    	{\footnotesize B} & {\footnotesize\em Prioridad Baja.}\\
    	{\footnotesize MB} & {\footnotesize\em Prioridad Muy Baja.}\\
		\hline
    \end{tabular} 
    \caption{Leyenda para los requerimientos funcionales.}
    \label{tbl:leyendaRF}
	\end{center}
\end{table}
=======
Este documento presenta la especificación técnica y conceptual del proyecto denominado ``Sistema inteligente de apoyo estudiantil para la Escuela Superior de Cómputo (ESCOM)''. Tiene como propósito detallar los requerimientos, objetivos y fundamentos del sistema, el cual propone una aplicación móvil con inteligencia artificial orientada a mejorar la experiencia académica y administrativa de los estudiantes.

El documento servirá como base de referencia para el desarrollo, validación y evaluación del prototipo final del sistema durante el ciclo escolar. De igual forma, constituye el entregable correspondiente al Trabajo Terminal 1 (TT1) y al documento de Requerimientos del Trabajo Terminal (RE-TT).

%---------------------------------------------------------
\section{Organización del contenido}

En este capítulo se introduce el contexto, los objetivos, la justificación y la metodología general del proyecto.  

En el capítulo \ref{cap:reqUsr} se describen los requerimientos del usuario y los casos de uso del sistema.  

En el capítulo \ref{cap:reqSist} se especifican los requerimientos funcionales y no funcionales, así como las trayectorias principales y alternativas asociadas a los casos de uso identificados.  

%---------------------------------------------------------
\section{Requerimientos Funcionales}

A continuación, se describen los requerimientos funcionales del sistema propuesto:

\begin{itemize}
	\item \textbf{RF1.} El sistema debe permitir al usuario consultar información sobre trámites escolares, profesores, clubs, becas, fechas, avisos y conferencias.
	\item \textbf{RF2.} El sistema debe proporcionar al usuario los medios de contacto de los profesores.
	\item \textbf{RF3.} El sistema debe permitir recibir notificaciones a los usuarios según sus preferencias.
	\item \textbf{RF4.} El sistema debe proporcionar un chatbot de ayuda para el usuario.
	\item \textbf{RF5.} El sistema debe proporcionar un mapa interactivo para facilitar la búsqueda de profesores, áreas administrativas y áreas comunes.
	\item \textbf{RF6.} El sistema debe permitir filtrar la visualización de áreas y profesores en el mapa interactivo.
\end{itemize}

%---------------------------------------------------------
\section{Requerimientos No Funcionales}

Los requerimientos no funcionales aseguran la calidad, disponibilidad y usabilidad del sistema:

\begin{itemize}
	\item \textbf{RNF1.} El sistema debe ser accesible desde un teléfono móvil o celular.
	\item \textbf{RNF2.} El sistema debe permitir una navegación fluida entre sus herramientas.
	\item \textbf{RNF3.} El sistema debe ser accesible y apto para los usuarios a quienes está dirigido.
	\item \textbf{RNF4.} El sistema debe ser intuitivo y dinámico para los usuarios.
	\item \textbf{RNF5.} El sistema debe contar con un diseño agradable para el usuario.
	\item \textbf{RNF6.} El sistema debe ser accesible en cualquier momento.
	\item \textbf{RNF7.} El sistema debe garantizar la seguridad en caso de que el usuario proporcione información personal.
	\item \textbf{RNF8.} El sistema debe permitir el acceso múltiple de usuarios de manera concurrente.
\end{itemize}

%---------------------------------------------------------
\section{Casos de Uso del Sistema}

\subsection{Caso de Uso CUR1: Consultar mapa interactivo}

\textbf{Actor:} Usuario \\
\textbf{Propósito:} Permitir al usuario consultar la ubicación de profesores y áreas, mostrando información relevante al seleccionar un área o profesor, como trámites, horarios y personal a cargo. \\
\textbf{Entradas:} Selección del mapa y filtros deseados desde el dispositivo móvil. \\
\textbf{Salidas:} Información mostrada sobre el área o profesor seleccionado. \\
\textbf{Errores:} 
\begin{itemize}
	\item MSG2: ``Sin conexión a internet.'' 
	\item MSG3: ``No se puede mostrar el mapa.''
\end{itemize}
\textbf{Trayectoria principal:}
\begin{enumerate}
	\item El usuario selecciona el botón de Mapa Interactivo.
	\item Selecciona el botón de aplicar filtros.
	\item Selecciona filtros de profesores o áreas.
	\item El sistema muestra el mapa con los filtros aplicados.
\end{enumerate}
\textbf{Trayectorias alternativas:}
\begin{itemize}
	\item A. El usuario solicita el mapa mediante el chatbot, y el sistema lo muestra.
\end{itemize}

%---------------------------------------------------------
\subsection{Caso de Uso CUR1.1: Filtrar profesores}

\textbf{Actor:} Usuario \\
\textbf{Propósito:} Permitir filtrar los profesores mostrados en el mapa por área, disponibilidad, carrera, nombre o unidad de aprendizaje. \\
\textbf{Entradas:} Selección de los filtros deseados desde la interfaz del mapa. \\
\textbf{Salidas:} Visualización del mapa con los filtros aplicados. \\
\textbf{Errores:} 
\begin{itemize}
	\item MSG2: ``Sin conexión a internet.'' 
	\item MSG3: ``No se puede mostrar el mapa.'' 
	\item MSG4: ``Profesor no encontrado.''
\end{itemize}
\textbf{Trayectoria principal:}
\begin{enumerate}
	\item El usuario selecciona la opción de filtrar profesores.
	\item Elige los filtros deseados (área, carrera, unidad de aprendizaje, disponibilidad).
	\item El sistema muestra el mapa actualizado con los filtros seleccionados.
\end{enumerate}

%---------------------------------------------------------
\subsection{Caso de Uso CUR1.2: Filtrar áreas}

\textbf{Actor:} Usuario \\
\textbf{Propósito:} Permitir filtrar las áreas mostradas en el mapa (administrativas, cafeterías, canchas, papelerías, etc.). \\
\textbf{Entradas:} Selección de los filtros desde el menú del mapa. \\
\textbf{Salidas:} Visualización de las áreas filtradas. \\
\textbf{Errores:}
\begin{itemize}
	\item MSG2: ``Sin conexión a internet.'' 
	\item MSG3: ``No se puede mostrar el mapa.'' 
	\item MSG4: ``Área no encontrada.''
\end{itemize}
\textbf{Trayectoria principal:}
\begin{enumerate}
	\item El usuario selecciona la opción de filtrar áreas.
	\item Elige los filtros deseados (cafeterías, clubs, baños, oficinas, etc.).
	\item El sistema muestra el mapa con las áreas seleccionadas.
\end{enumerate}

%---------------------------------------------------------
\subsection{Caso de Uso CUR2: Consultar profesor}

\textbf{Actor:} Usuario \\
\textbf{Propósito:} Consultar información sobre un profesor (ubicación, horario, medios de contacto, áreas de interés). \\
\textbf{Entradas:} Solicitud del usuario desde la interfaz o mediante el chatbot. \\
\textbf{Salidas:} Información del profesor seleccionada. \\
\textbf{Error:} MSG5: ``Profesor no encontrado.'' \\
\textbf{Trayectoria principal:}
\begin{enumerate}
	\item El usuario selecciona la opción “Consultar profesor”.
	\item El sistema muestra la información relevante del profesor.
\end{enumerate}

%---------------------------------------------------------
\subsection{Caso de Uso CUR3: Utilizar chatbot}

\textbf{Actor:} Usuario \\
\textbf{Propósito:} Facilitar al usuario la obtención de información sobre ESCOM mediante un chatbot inteligente. \\
\textbf{Entradas:} Pregunta o mensaje ingresado por el usuario. \\
\textbf{Salidas:} Respuesta generada por el chatbot. \\
\textbf{Errores:}
\begin{itemize}
	\item MSG2: ``Sin conexión a internet.'' 
	\item MSG6: ``Lo siento, solo puedo ayudarte con preguntas relacionadas a ESCOM.''
\end{itemize}
\textbf{Trayectoria principal:}
\begin{enumerate}
	\item El usuario accede al chatbot.
	\item Realiza una pregunta o selecciona una opción predefinida.
	\item El sistema muestra la respuesta correspondiente.
\end{enumerate}

%---------------------------------------------------------
\subsection{Caso de Uso CUR4: Consultar trámites}

\textbf{Actor:} Usuario \\
\textbf{Propósito:} Proporcionar al usuario la información sobre algún trámite de su interés. \\
\textbf{Entradas:} Solicitud del usuario desde la opción “Consultar Trámite”. \\
\textbf{Salidas:} Información del trámite seleccionado. \\
\textbf{Trayectoria principal:}
\begin{enumerate}
	\item El usuario selecciona la opción “Consultar Trámite”.
	\item El sistema muestra la información del trámite solicitado.
\end{enumerate}

%---------------------------------------------------------
\section{Leyenda de abreviaciones utilizadas}

\begin{table}[hbtp!]
	\begin{center}
		\begin{tabular}{|r l|}
			\hline
			{\footnotesize Id} & {\footnotesize\em Identificador del requerimiento.}\\
			{\footnotesize Pri.} & {\footnotesize\em Prioridad asignada (MA, A, M, B, MB).}\\
			{\footnotesize Ref.} & {\footnotesize\em Referencia a requerimientos de usuario (si aplica).}\\
			{\footnotesize MA} & {\footnotesize\em Prioridad Muy Alta.}\\
			{\footnotesize A} & {\footnotesize\em Prioridad Alta.}\\
			{\footnotesize M} & {\footnotesize\em Prioridad Media.}\\
			{\footnotesize B} & {\footnotesize\em Prioridad Baja.}\\
			{\footnotesize MB} & {\footnotesize\em Prioridad Muy Baja.}\\
			\hline
		\end{tabular} 
		\caption{Leyenda para las abreviaciones usadas en los requerimientos.}
		\label{tbl:leyendaRS}
	\end{center}
\end{table}
>>>>>>> fa97d892322beb1d619a1051adfc14a90fd5e6af
